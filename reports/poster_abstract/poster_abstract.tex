
\documentclass[10pt,twoside]{article}
\usepackage{url}

\newcommand{\doctitle}{%
Approximations for the Feedback Vertex Set Problem}

\pagestyle{myheadings}
\markboth{\hfill\doctitle}{\doctitle\hfill}

\bibliographystyle{siam}

\addtolength{\textwidth}{1.00in}
\addtolength{\textheight}{1.00in}
\addtolength{\evensidemargin}{-1.00in}
\addtolength{\oddsidemargin}{-0.00in}
\addtolength{\topmargin}{-.50in}

\hyphenation{in-de-pen-dent}

%\title{\textbf{\doctitle}\\
\title{\textbf{ Binary Decompilation to LLVM IR}}

\author{Sandeep Dasgupta\thanks{Electronic address: \texttt{sdasgup3@illinois.edu}}
\qquad Vikram Adve\thanks{Electronic address: \texttt{vadve@illinois.edu}}
} 

\begin{document}

\thispagestyle{empty}

\maketitle

Analyzing and optimizing programs from their executable has a long history of
research pertaining to various applications including  security vulnerability
analysis, untrusted code analysis, malware analysis, program testing, and
binary optimizations. 

This work is a step towards the same broader objective by decompiling the input binary
into an intermediate form ( IR ) of LLVM, which is a widely-used compiler
infrastructure.  The main challenge of the work involves extracting the
variable (both scalar and aggregate) and type information from the input binary
into a fully functional IR.  For this we have used a publicly available tool
called McSema \cite{Mcsema}  which can convert x86
machine code to functional LLVM IR. McSema support translation of x86 machine
code, including integer, floating point, and SSE instructions. One of the
downside of McSema recovered IR is that the variable (scalar/aggregate)
and type information is missing. The current work is about recovering those.

As an initial step, we have developed tools \cite{SourceMapper} for better
debugging of McSema generated IR.  Notables are ``source mapper'', which maps
source ( input binary ) information to generated LLVM IR, and a LLVM backward
slicer. Also we have identified optimization oportunities, like scalar
replacement of aggregates, in the design of McSema generated IR to improve its
quality. 

Mcsema uses a big flat array to model the runtime process stack i.e. all the
reads/writes made by a binary on its runtime stack are modeled into this array.
The first step towards our goal is to identify variables in this array and
promote them as separate symbols which requires deconstructing this global 
array, that Mcsema shares between all the procedure, into per procedure array
which is used to model the stack frame of that procedure.  Such stack deconstruction is important because 
doing symbol promotion right on the global array could be very conservative because
 an indirect write made by a different procedure may prevent
symbol promotion in the current procedure.  We have implemented a
transformation pass which can do the stack deconstruction and also implemented variable recovery
and symbol promotion as described in \cite{PLDI2013,Eurosys2013}

The next step is to infer the type of the recovered variables. 
We are developing a prototype model for type inferencing based on \cite{PLDI2013,PLDI2016}. 

\nocite{*}
\bibliography{poster_abstract}

\end{document}
